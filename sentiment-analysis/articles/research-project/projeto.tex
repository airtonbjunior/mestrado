\documentclass[a4paper,11pt]{article}
\usepackage[top=2cm, bottom=2cm, left=2cm, right=2cm]{geometry}
\usepackage[utf8]{inputenc} % O beamer já carregou
\usepackage[T1]{fontenc}
\usepackage[english,brazil]{babel}
\usepackage{graphicx}
\usepackage{enumerate}
\renewcommand*\rmdefault{ppl}
\usepackage{titlesec}
\usepackage{pgfgantt}
\usepackage{framed}
\newcommand\sectionbreak{\clearpage}
\usepackage{fancyhdr}
\pagestyle{fancy}
\usepackage{lastpage}
\setlength{\headheight}{26pt} 
\fancypagestyle{plain}{%
  \renewcommand{\headrulewidth}{0pt}%
  \fancyhf{}%
  \fancyfoot[L]{Metodologia científica - UFG}%
  \fancyfoot[R]{\footnotesize Página \thepage\ de \pageref{LastPage}}%
}


%head and foot note
\lhead{\textbf{Mestrado em Ciência da Computação} \\\textbf{Projeto de pesquisa}}
\chead{}
\rhead{\textbf{Metodologia Científica}\\ \textit{Airton Bordin Junior}}
\lfoot{}
\cfoot{}
\rfoot{\footnotesize Página \thepage\ de \pageref{LastPage}}

\renewcommand{\headrulewidth}{0.1pt}
\renewcommand{\footrulewidth}{0.1pt}
\makeindex
\begin{document}

\begin{titlepage}

\newcommand{\HRule}{\rule{\linewidth}{0.5mm}} % Defines a new command for the horizontal lines, change thickness here

\begin{center} % Center everything on the page
 
%----------------------------------------------------------------------------------------
%	HEADING SECTIONS
%----------------------------------------------------------------------------------------

\begin{figure}
\centering
\begin{minipage}{.5\textwidth}
  \centering
  \includegraphics[width=.4\linewidth,scale=1.3]{logo-uel}
\end{minipage}%
\begin{minipage}{.5\textwidth}
  \centering
  \includegraphics[scale=0.2]{logo-ufg}
%    \includegraphics[width=.4\linewidth,scale=0.1]{logo-ufg}

\end{minipage}
\end{figure}

%\includegraphics[scale=0.5]{logo-ufg}\\[1.0cm] % Include a department/university logo - this will require the graphicx package

%\textsc{\LARGE Universidade Federal de Goi�s}\\[1.5cm] % Name of your university/college

%\textsc{\Large \textbf{Instituto de Computa��o}}\\[0.5cm] % Major heading such as course name
\textsc{\large \textbf{Mestrado em Ciência da Computação}}\\[0.5cm]
\textsc{\large \textbf{Disciplina de Metodologia Científica}}\\[1.5cm] % Minor heading such as course title

%----------------------------------------------------------------------------------------
%	TITLE SECTION
%----------------------------------------------------------------------------------------

\HRule \\[0.4cm]
{ \huge \bfseries Expansão automatizada de Léxicos para a Análise de Sentimentos por meio de programação evolucionária}\\[0.4cm] % Title of your document
\HRule \\[1.5cm]

%----------------------------------------------------------------------------------------
%	AUTHOR SECTION
%----------------------------------------------------------------------------------------

\begin{framed}
\emph{Autor:} Airton Bordin Junior
\par
\emph{Orientador:} Nádia Félix Felipe da Silva
\par
\emph{Coorientador:} Celso Gonçalves Camilo Junior
\end{framed}

\vspace*{3cm}

%----------------------------------------------------------------------------------------
%	DATE SECTION
%----------------------------------------------------------------------------------------

{\large \today}\\ % Date, change the \today to a set date if you want to be precise

\end{center}

%----------------------------------------------------------------------------------------

\vfill % Fill the rest of the page with whitespace

\end{titlepage}

\section{Apresentação}
Airton Bordin Junior, bacharel em Ciência da Computação. Cursou os primeiros 3 anos do curso na Universidade Estadual do Oeste do Paraná (UNIOESTE) campus Foz do Iguaçu, e finalizou a graduação na faculdade Anglo Americano, na mesma cidade, no ano de 2011.

Possui, também, graduação em Gestão Pública, pelo Instituto Federal de Santa Catarina, cursado por meio da UaB, no campus de Foz do Iguaçu.

Após a graduação, cursou especialização em Redes de Computadores pela Universidade Federal Tecnológica do Paraná, campus de Cornélio Procópio e MBA em Gerenciamento de Projetos pelo Centro Universitário Dinâmica das Cataratas, em Foz do Iguaçu.

Atua profissionalmente na área da computação desde 2010. Trabalhou como desenvolvedor de software, testador e, por fim, como Analista de Sistemas no Parque Tecnológico Itaipu, responsável pela área de TI do projeto de Segurança de Barragens

Lecionou 2 semestres no curso Técnico em Informática Para Internet do Pronatec, atuando nas disciplinas de Sistemas Operacionais e Segurança de Sistemas, e 1 semestre no curso de Ciência da Computação em uma faculdade local, lecionando as disciplinas de Processamento de Imagens e Sistemas Inteligentes.

Sempre quis continuar estudando, e o mestrado era um objetivo a ser atingido. Por conta da falta de oportunidades na cidade onde morava (Foz do Iguaçu), esta meta teve que ser adiada. Hoje, tem a oportunidade e a honra de participar como aluno regular do programa de mestrado pela Universidade Federal de Goiás, na linha de pesquisa de Inteligência Computacional.

Prentende aprofundar o trabalho na área de Mineração de Opiniões, também chamada de Análise de Sentimentos, mais precisamente na criação e expansão automatizada de dicionários léxicos, alinhado com trabalhos em andamento de alguns professores da Universidade. O interesse em pesquisar esse assunto vem crescendo nos últimos anos, principalmente com o aumento da produção de conteúdo na WEB, e apresenta-se como um desafio interessante e atual e que pode trazer benefícios para diversas outras áreas de conhecimento como, por exemplo, o setor de saúde.

\section{Resumo}
Deve ter aproximadamente 300 palavras. Alem disto, deve ter uma descricao breve de todo o projeto, atendendo a estas quatro areas:

\begin{itemize}
\item{O que voce vai fazer? (o problema)}
\item{Como sera feito (metodologia)}
\item{Resultados esperados? (apenas os mais relevantes)}
\item{Qual a importancia destes? (Conclusoes / recomendacoes)}
\end{itemize}

\section{Introdução}

A Mineração de Opiniões, também chamada de Análise de Opiniões ou Análise de Sentimentos, é uma linha de pesquisa abrangente e que vem sendo tema de diversos trabalhos nos últimos anos. Como observado em \cite{liu2010multifaceted}, esse crescente interesse sobre o assunto ocorre principalmente devido ao aumento no número de usuários de Internet e o consequente crescimento da produção de conteúdo independente na rede, como opiniões, avaliações, entre outros. 

Essa área de estudo tem como principal desafio a Análise de Opiniões, descritas em linguagem natural, para a identificação da polaridade implícita ou explícita no texto. Essa polaridade é, na maior parte das vezes, identificada como uma escala de pontuação de sua característica positiva, negativa ou neutra.

Uma das principais técnicas para aumentar a acurácia a Análise de Sentimentos é a utilização de Dicionários de Dados. Esses dicionários contêm palavras previamente avaliadas por especialistas humanos, principalmente quanto à sua polaridade. Neste contexto, esse conjunto de palavras, juntamente com suas polaridades, é chamado de Dicionário Léxico ou Dicionário de Sentimentos. 

Porém, é evidente a limitação inerente à estratégia de utilização do Dicionário Léxico - a própria lista de palavras disponíveis. Esse fato muitas vezes limita a realização de uma análise mais profunda sobre determinado contexto. Nesse sentido, um dos principais desafios na área de Mineração de Opiniões é a criação e ampliação do Dicionário Léxico de forma automatizada, tema central do presente trabalho. Grande parte desses dicionários são construídos de forma manual, fato que caracteriza uma limitação óbvia para a maior parte dos contextos e domínios, como observado em \cite{duwairi2015detecting}. 

Consciente dessa limitação, a ideia principal do presente trabalho é a criação de um processo automatizado de expansão de Dicionário Léxico, sensível a domínios específicos, fazendo uso de técnicas de algoritmos bioinspirados da classe de Algoritmos Evolucionários, mais precisamente Programação Evolucionária. Resumidamente, o processo fará a adequação das polaridades das palavras contidas no dicionário de forma a maximizar a corretude das avaliações. Para avaliar a taxa de erro de cada solução parcial gerada, serão utilizados conjuntos conhecidos de documentos previamente avaliados por especialistas humanos, de forma a compará-los com a solução gerada pelo sistema. Para a avaliação, sistemas de classificação de sentimentos disponíveis serão utilizados.

Devido à característica intrínseca do próprio problema, serão utilizadas na pesquisa bases de textos em inglês. Ao mesmo tempo, muitas técnicas utilizadas durante o trabalho também poderão ser utilizadas para resolver problemas em português, com as devidas alterações. 

Espera-se que esse processo, bem como os Dicionários Léxicos por ele gerado, possam ser utilizados como entrada de processos de avaliação em diversas áreas de Mineração de Opiniões como, por exemplo, Análise de Sentimentos em redes sociais. Algumas dessas pesquisas são realizadas na própria instituição, apoiando, assim, o trabalho de outros pesquisadores. Além disso, devido ao caráter automatizado dessa solução proposta, o mesmo processo poderá ser utilizado, avaliado e melhorado para outras situações, contextos e idiomas.

Por fim, a pesquisa e a utilização de diversas técnicas de PNL e expansão automatizada de Dicionário Léxico poderão servir como um \emph{benchmark} dos principais métodos e classificadores, auxiliando na escolha de ferramentas e abordagens para trabalhos futuros em contextos específicos.

Para apoiar a clareza e desenvolvimento da proposta, o presente documento está estruturado da seguinte forma: o próximo capítulo tratará da descrição do problema, apresentando as principais limitações e dificuldades no contexto de Análise de Sentimentos e Dicionários Léxicos. O capítulo \ref{sec:obj} tratará dos objetivos gerais e específicos. A revisão bibliográfica, apresentada no capítulo \ref{sec:bibl}, tem por objetivo o embasamento teórico para apoiar nas soluções propostas, apresentando o estado da arte sobre o assunto, bem como definindo conceitos fundamentais para entender e trabalhar com a Análise de Sentimentos. O capítulo \ref{sec:impact} apresenta o impacto científico da solução e suas possíveis contribuções para a área. A metodologia, descrita no capítulo \ref{sec:met}, descreve a forma como serão desenvolvidas cada uma das etapas do processo, seguida do capítulo com uma previsão de cronograma do trabalho. Resultados esperados são descritos no capítulo \ref{sec:result}, seguidos da identificação dos colaboradores e participantes do projeto e, por fim, as referências bibliográficas utilizadas na proposta.


\section{Descrição do Problema}
Como discutido na seção anterior, uma das principais técnicas utilizadas para a Análise de Sentimentos faz uso de um Dicionário Léxico de palavras e suas polaridades, geralmente classificadas como positiva, negativa ou neutra. Essa abordagem, apesar de trazer benefícios ao processo de Mineração de Opiniões, possui algumas limitações. A principal delas está justamente ligada à disponibilidade das palavras, bem como sua correta polaridade. Outra dificuldade encontrada no uso de dicionários vem do fato que, em diferentes domínios, uma palavra pode ter um significado e, até mesmo, uma força sentimental diferente. A palavra "câncer", por exemplo, em um contexto médico, pode não ter uma conotação negativa (muitas vezes é uma palavra neutra), diferente de outros contextos.

Construir Dicionários Léxicos para domínios específicos é mais complexo que a construção de conjunto de palavras independentes de contexto, como cita \cite{Kanayama2006}.

Buscando resolver esses problemas - falta de palavras no Dicionário Léxico e criação de Dicionários para contextos específicos - muitas técnicas foram desenvolvidas nos últimos anos. A maior parte delas utiliza a estrutura sintática dos textos como forma de tentar encontrar a polaridade mais adequada a um conjunto de palavras. Dentre as principais técnicas podemos citar o Pointwise Mutual Information, apresentado por \cite{Turney2002}, coerência do contexto, discutido em \cite{Kanayama2006}, entre outros.

Uma limitação às soluções anteriores vem do fato da dificuldade inerente em trabalhar com dados não estruturados, neste contexto, linguagem natural. Esse fato é agravado quando estamos trabalhando com textos em redes sociais, por possuírem um caráter informal, contendo abreviações, gírias, trocadílhos, etc. Portanto, não é trivial expandir um Dicionário Léxico, principalmente para domínios específicos, fazendo uso das estruturas sintáticas e semânticas dos textos.

Determinar a polaridade correta da palavra, para cada contexto, portanto, apresenta-se como um desafio para as pesquisas na área. Mesmo as principais técnicas utilizadas demandam uma validação manual considerável, e em alguns contextos, como política, não atingem resultados satisfatórios.

Testar todas as possibilidades para a criação de um Dicionário Léxico perfeito, ou com o menor erro possível, também torna-se inviável, pois demandaria a resolução de um problema combinatório com muitas variáveis, o que demandaria um tempo exponencial de processamento, caracterizando-se como um problema NP-completo, ou seja, não solucionável em tempo polinomial.

Uma forma eficiente de resolver essa classe de problemas é a utilização de algoritmos bioinspirados. Por sua característica inerentemente paralela, esse tipo de abordagem consegue encontrar soluções ótimas para problemas complexos com grandes espaços de busca. Apesar de, geralmente, iniciar a busca da solução de forma aleatória, processos de modificação e adequação do algoritmo (nesse contexto chamado de operadores genéticos) permitem que as melhores soluções sejam selecionadas e evoluídas, de forma a maximizar o resultado final do processo.

Para testar a acurácia dos resultados desse processo, é necessário realizar testes em classificadores de sentimentos disponíveis, bem como a utilização de conjuntos de opiniões previamente avaliados por especialistas humanos, para comparar com os resultados da solução proposta.

\section{Objetivos}
\label{sec:obj}
\subsection{Objetivo geral}
O presente trabalho tem por objetivo a criação de um sistema para a expansão de um Dicionário Léxico, contendo palavras e suas respectivas orientações semânticas (positiva, negativa, neutra) para um determinado contexto, fazendo uso de técnicas de algoritmos bioinspirados, mais precisamente Programação Evolutiva. A criação de um Léxico abrangente e específico para um determinado contexto é um desafio para a área de Análise de Sentimentos, e fundamental para o correto funcionamento de todo o processo de análise dos dados.

\subsection{Objetivos especificos}

\begin{enumerate}
\item Criação de um sistema para a expansão automatizada de dicionário léxico para domínios específicos;
\item Criação de dicionários léxicos consolidados para domínios específicos, prontos para serem utilizados por sistemas de Análise de Sentimentos;
\item Estudo comparativo da técnica proposta com outras técnicas da literatura, de forma a apoiar a evolução de soluções existentes;
\item Publicação de trabalhos sobre o assunto de forma a expandir o conhecimento sobre a utilização de algoritmos bioinspirados na área de Análise de Sentimentos.

\end{enumerate}

\section{Revisão bibliográfica}
\label{sec:bibl}
O aumento considerável na quantidade de conteúdo disponível na WEB nos últimos anos tornou grandes e valiosas bases de dados e informações disponíveis facilmente para todos os interessados em analisá-las, como bem afirma \cite{kdir16}. A crescente utilização de redes sociais e o consequente aumento no número de compartilhamento de opiniões pessoais motivaram o interesse crescente na área de Análise de Sentimentos, também chamada de Mineração de Senimentos ou Opiniões. 

Uma visão multifacetada da área de pesquisa é apresentada por \cite{liu2010multifaceted}, conceitualizando o problema e propondo uma forma estruturada de organização dos dados não estruturados, característica instínseca dos textos em linguagem natural, objeto de entrada da pesquisa. A definição de opinião como uma quíntupla (entidade, aspecto da entidade, sentimento, autor e tempo) é utilizada em grande parte dos trabalhos na área, caracterizando-se, portanto, como elemento fundamental nas pesquisas sobre o assunto. Visão geral sobre o tema e principais desafios e técnicas são vistos também em \cite{mohammad2016challenges}, \cite{ghaleb2016survey}, \cite{kdir16}, \cite{taboada2011lexicon}, \cite{bandhakavi2016lexicon}, entre outros trabalhos.

Uma das formas mais comuns para realizar a Análise de Sentimentos é por meio da utilização de um Dicionário Léxico (algumas vezes chamado de Dicionário de Sentimentos), um conjunto de palavras e suas orientações semânticas (também chamadas de polaridades), frequentemente representadas como positiva, negativa ou neutra. A obtenção de um Dicionário consistente é essencial para uma correta Mineração de Sentimentos, conforme enfatiza \cite{kdir16}. 

Em \cite{kdir16}, observamos que existem, basicamente, 3 formas de criação e expansão de um Dicionário Léxico: manual -  processo realizado por especialistas humanos que analisam cada palavra, atribuindo uma Orientação Semântica para cada uma delas -  e duas formas (semi) automatizadas: baseada em Dicionário e baseada em Corpus. Frequentemente, essas técnicas são utilizadas em conjunto, principalmente a validação manual de Dicionários criados de forma automatizada. Criações de Dicionários utilizando somente abordagem manual, por sua característica limitante, são menos utilizadas, e não serão abordadas de forma mais aprofundada no decorrer deste trabalho.

No contexto de prognóstico automatizado de Orientação Semântica de palavras, um dos primeiros trabalhos apresentados foi \cite{Hatzivassiloglou}, focando na previsão de polaridade de adjetivos.

\cite{Turney2002} apresenta uma abordagem de expansão léxica fazendo uso da técnica de Pointwise Mutual Infomation (PMI), com o objetivo de calcular a co-ocorrência de palavras e, com isso, comparar a polaridade de novas palavras com outras já conhecidas. Nesse trabalho, amplamente referenciado por outras pesquisas, o autor compara o conjunto de palavras de Orientação Semântica desconhecida com as palavras "excellent" e "poor", representado Orientações Semânticas positiva e negativa, respectivamente. Essas palavras previamente conhecidas utilizadas como base para a expansão do Dicionário são chamadas de palavras semente (\emph{seed words}, em inglês). Como exemplo de trabalhos que utilizam o PMI para a criação e expansão do Dicionário Léxico podemos citar \cite{becker2013}, \cite{Zhou2014}, \cite{Pinto2007}. \cite{Pantel2006}, entre outros.

A maior parte das estratégias de criação de dicionários léxicos utiliza como base de palavras semente o banco de dados \emph{WordNet} - disponível em https://wordnet.princeton.edu/ - que fornece uma lista de palavras, sua polaridade e seus sinônimos. Importante destacar, também, que as bases utilizadas nos trabalhos citados consideram palavras no idioma inglês. Mesmo alguns trabalhos que abordaram idiomas diferentes fizeram uso dessas bases por meio de um processo de tradução automatizada.

\cite{taboada2011lexicon} apresenta uma abordagem baseada em Léxico combinada com uma verificação manual. Esse trabalho apresenta o SO-CAL (Semantic Orientation Calculator), que usa lista de palavras já consolidadas para a geração de dicionários com novas entradas e suas polaridades de forma não supervisionada. Durante a descrição do trabalho, apresenta conceitos de intensificação e negação, amplamente utilizadas nas técnicas de geração de novos Dicionários. Apesar de ser feita de forma automática, o autor utilizou uma etapa de verificação humana para a validação da consistência das palavras geradas pela técnica, fazendo uso de um serviço de\emph{ Mechanical Turk} da Amazon. 

Na mesma linha, \cite{eisenstein2016unsupervised} e \cite{bandhakavi2016lexicon} apresentam procedimentos para apoiar a Análise de Sentimentos. O primeiro apresenta uma abordagem usando a técnica de \emph{Naive Bayes} para a classificação dos aspectos e cita problemas de estimativas de palavras e avaliação dos léxicos criados. O segundo faz uma comparação de algumas técnicas de avaliação em 4 conjuntos de dados diferentes, apresentando uma análise quantitativa do mesmo. Abordagens e comparações semelhantes, com algumas modificações no domínio e no idioma do problema abordado, podem ser vistos em \cite{khoo2017lexicon}, \cite{asghar2014review} e \cite{ding2008holistic}.

A maior parte dos trabalhos citados trata de todo o processo de Análise de Sentimentos. A criação automatizada de Dicionários Léxicos, objetivo principal do presente trabalho, é tratada de forma central em \cite{widdows2002graph} e \cite{duwairi2015detecting}. O primeiro utiliza uma estratégia de criação e análise de uma estrutura de grafos, por meio uma base padronizada de palavras semente, que contém diversas entradas previamente avaliadas em suas polaridades e, também, a descrição de seus sinônimos. Apesar de fazer uma abordagem focada em substantivos, que representam os vértices do grafo, a ideia principal pode ser utilizada em outras estratégias de geração léxica automatizada que incorporem verbos, adjetivos, entre outros. \cite{duwairi2015detecting} dá uma visão geral da criação de um dicionário de palavras, usando como base \emph{tweets} em árabe. Importante destaque desse último foi a inclusão de \emph{emoticons} na análise, característica amplamente utilizada, principalmente, em escritas informais na Internet.

ATÉ AQUI TÁ BOM


\cite{kaji} aborda uma estratégia de criação de dicionários analisando uma coleção de páginas HTML. Apesar de trabalhar com o idioma japonês, a técnica pode ser adequada para outros idiomas.

COLOCAR AQUI ALGUMA COISA SOBRE CLASSIFICADORES
	
Uma abordagem qu

\section{Impacto Científico}
\label{sec:impact}
A técnica de Análise de Sentimento por meio de um Dicionário Léxico é uma das mais utilizadas na literatura. Mostra-se, portanto, essencial a obtenção de um conjunto de palavras consolidado, juntamente com as orientações semânticas respectivas. 
Uma palavra pode ter um significado e, consequentemente, uma polaridade diferente, dependendo do contexto ao qual está inserido. 

Um conjunto de palavras e polaridades inadequadas leva a análises inconsistentes, prejudicando o resultado final do sistema.

A solução proposta neste trabalho criará, de forma automatizada, Léxicos para diferentes domínios, que poderão servir como entrada para diversos classificadores e sistemsas de análise de sentimentos. Além disso, a técnica pode ser utilizada em outros idiomas, de forma a suprir uma carência de dicionários consistentes em linguagens pouco conhecidas.
O conjunto de palavras gerado pela solução proposta poderá ser utilizado como \emph{benchmark} para outros trabalhos na área, bem como ser expandido com outras técnicas adequadas.

Técnicas de algoritmos bioinspirados, da classe de algoritmos evolucionários, serão utilizadas para a resolução do problema. A intersecção dessas áreas de conhecimento foi pouco explorada até o momento na literatura e em trabalhos realizados, caracterizando, portanto, uma nova abordagem para vistas às possíveis soluções. A utilização da abordagem proposta pode incentivar a utilização de outros métodos bioinspirados, apoiando, portanto, uma alternativa às soluções mais utilizadas, baseadas em análise sintática e semântica.

\section{Metodologia}
\label{sec:met}
Para atingir os objetivos da pesquisa, buscas na literatura serão realizadas de forma a entender o estado da arte sobre o assunto, principais soluções e abordagens utilizadas. Esses dados serão utilizados como embasamento teórico para o desenvolvimento do trabalho. 

Além disso, será feita uma pesquisa das principais ferramentas, preferencialmente livres e \emph{open source} para a utilização nos testes da solução. Descrições mais detalhadas de cada etapa podem ser encontradas nos próximos subcapítulos.

Um sistema de expansão de Dicionário Léxico será implementado e testado com algumas entradas avaliadas previamente por especialistas humanos.

Após as fases de projeto, desenvolvimento e teste da solução, dados serão coletados para a criação de indicadores sobre o sistema, bem como comparações com soluções disponíveis na literatura.

\subsection{Etapas}
\begin{enumerate}[D1.]
\item{\textbf{Revisao bibliografica:} Nesta etapa do trabalho sera feita uma revisao bibliografica com vistas a identificar o estado da arte do problema que esta sendo proposto. Importante registrar que esta revisao bibliografica seguira os moldes propostos por \cite{Kitchenham2004}}. Serao consultadas as bases de dados do [COLOCAR BASES AQUI] \emph{Portal da Capes}, \emph{IEEEXplore} e \emph{ACM Digital Library}.

\item{\textbf{Estudo dos principais classificadores de sentimentos:}}
Nesta etapa será feita uma pesquisa sobre os principais classificadores disponíveis, preferencialmente livres e \emph{open source}, utilizados para a análise de sentimentos, e que permitem a manipulação de seu dicionário. O objetivo principal desse passo é selecionar ferramentas que proporcionarão dados comparativos para teste da solução proposta.

\item{\textbf{Análise dos principais léxicos disponíveis:}}
Busca pelos principais conjuntos de palavras, e suas respectivas polaridades, disponíveis para utilização. Esses Dicionários Léxicos servirão como base e material de testes para a solução.

\item{\textbf{Recuperação das principais bases de opiniões anotadas disponíveis:}}
Busca e recuperação de bases de opiniões anotadas e consistentes, representando resultados confiáveis e corretos. Essas bases, já revisadas por especialistas humanos, servirão como parâmetro de corretude da solução proposta, bem como serão utilizadas para o cálculo de erro dos resultados obtidos.

\item{\textbf{Implementação da solução:}}
Implementação da solução proposta, com o objetivo da expansão de um léxico, sensível a um domínio específico, que servirá como entrada para um classificador utilizado em processos de análise de sentimentos.

\item{\textbf{Teste da solução com os classificadores selecionados:}}
Teste dos resultados fazendo uso dos classificadores selecionados anteriormente, de forma a obter um resultado satisfatório, minimizando a taxa de erros ao comparar com resultados consolidados e previamente revisados.

\item{\textbf{Levantamento dos dados de testes e relatórios:}}
Levantamento dos dados da utilização da solução, fazendo uso dos classificadores selecionados, e fazendo a comparação com outros sistemas e soluções disponíveis na literatura. Essa etapa fará a classificação e organização dos resultados, de forma a facilitar a visualização, entendimento, e auxiliar na tomada de decisões sobre o projeto.
\end{enumerate}

\subsection{Marcos fisicos}
\begin{enumerate}[D1.]
\item{Documento com a revisao bibliografica.}
\item{Lista dos principais classificadores.}
\item{Lista dos principais léxicos.}
\item{Bases de opiniões recuperadas.}
\end{enumerate}

\section{Cronograma de trabalho}
\label{sec:crono}

\begin{ganttchart}{1}{24}
\gantttitle{2017}{10}\gantttitle{2018}{12}\gantttitle{2019}{2} \\
\gantttitlelist{3,...,12}{1}\gantttitlelist{1,...,12}{1}\gantttitlelist{1,...,2}{1} \\
\ganttgroup{Disciplinas}{1}{9} \\
\ganttmilestone{Fim do $1^o$ semestre}{5} \ganttnewline
\ganttmilestone{Fim do $2^o$ semestre}{10} \ganttnewline
\ganttgroup{Dissertacao}{1}{23} \\
\ganttbar{$D_1$}{1}{4} \\
\ganttbar{$D_2$}{3}{4} \\
\ganttbar{$D_3$}{4}{6} \\
\ganttbar{$D_4$}{5}{7} \\
\ganttbar{$D_5$}{6}{14} \\
\ganttbar{$D_6$}{12}{16} \\
\ganttbar{$D_7$}{17}{20} \\
%\ganttlink{elem5}{elem6}
\ganttmilestone{Relatorio 1}{4} \ganttnewline
\ganttmilestone{Relatorio 2}{9} \ganttnewline
\ganttmilestone{Relatorio 3}{14} \ganttnewline
\ganttlink{elem9}{elem10}
\ganttlink{elem5}{elem9}
\ganttmilestone{Qualificacao}{17}\ganttnewline
\ganttmilestone{Defesa}{19}
\end{ganttchart}

\begin{center}
\large \textbf{Legenda}
\end{center}

\begin{enumerate}[D1.]
\item{Revisao bibliografica.}
\item{Estudo dos principais classificadores de sentimentos.}
\item{Análise dos principais léxicos disponíveis.}
\item{Recuperação das principais bases de opiniões anotadas disponíveis.}
\item{Implementação da solução.}
\item{Teste da solução com os classificadores selecionados.}
\item{Levantamento dos dados de testes e relatórios.}
\end{enumerate}

\section{Resultados Esperados}
\label{sec:result}
Espera-se, com o presente trabalho, a criação de um processo automatizado de expansão de léxico dependente de domínio, fazendo uso de técnicas de algoritmos evolucionários. Nesse sentido, expansão significa tanto a criação e definição da orientação semântica de novas palavas, bem como a alteração das polaridades das palavras já existentes para um valor mais adequado ao domínio que trata o processo.
Pela característica genérica da solução, a criação de diversos léxicos para vários domínios diferentes é limitada tão somente à escolha dos contextos específicos e à disponibilidade de dados anotados para teste da solução.
Podemos citar, também, uma possível melhoria em algumas técnicas de Análise de Sentimentos que fazem uso de léxicos padrão, contribuindo assim para a evolução de outros sistemas de Mineração de Opiniões que usam a estratégia de dicionário.
Os resultados parciais e finais do trabalho serão descritos em artigos científicos que serão submetidos à eventos na área, de forma a compartilhar o conhecimento e avanços alcançados pela técnica proposta.

\subsection{Algoritmos} % sugestao 
Será desenvolvido um algoritmo que criará e/ou ampliará, de forma automatizada, um léxico para um domínio específico que será utilizado como entrada para um sistema classificador de Análise de Sentimentos.
Esse software fará uso de técnicas de algoritmos bioinspirados, mais precisamente Programação Evolucionária, para a atribuição de valores sentimentais para cada palavra, de forma a maximizar a taxa de acerto ao ser processado por um classificador existente.
Ao passo que o algoritmo é independente de domínio, pode ser utilizado, desde que haja dados de testes suficientes, para qualquer contexto desejado.

\section{Identificacao dos Participantes e Colaboradores}
O presente trabalho é parte de uma pesquisa maior, realizada na UFG, que estuda a Análise de Sentimentos em todas as suas etapas.

Um projeto anterior, SentiHealth \cite{Rodrigues2016}, será utilizado como bases para testes, principalmente referente ao módulo de classificação. Além disso, o dicionário utilizado no sistema será a entrada da solução proposta servindo como entrada, portanto, do algoritmo que fará a expansão do Léxico.

Durante o trabalho, a colaboração mútua entre os participantes do grupo de pesquisa em Análise de Sentimentos será fundamental. Nesse sentido, há alunos de graduação (iniciação científica) e mestrado, que terão papel fundamental para o sucesso do tema proposto.

Por fim, mas não menos importante, a colaboração do orientador e coorientador do trabalho, ambos membros do grupo de pesquisa, será muito importante para o correto andamento do trabalho e para que os objetivos sejam atingidos.
\section{Referencias bibliograficas}
\bibliographystyle{apalike}
\bibliography{projeto}
\end{document}