\documentclass[a4paper,11pt]{article}
\usepackage[top=2cm, bottom=2cm, left=2cm, right=2cm]{geometry}
\usepackage[utf8]{inputenc} % O beamer já carregou
\usepackage[T1]{fontenc}
\usepackage[english,brazil]{babel}
\usepackage{graphicx}
\usepackage{enumerate}
\renewcommand*\rmdefault{ppl}
\usepackage{titlesec}
\usepackage{pgfgantt}
\usepackage{framed}
\newcommand\sectionbreak{\clearpage}
\usepackage{fancyhdr}
\pagestyle{fancy}
\usepackage{lastpage}
\setlength{\headheight}{26pt} 
\fancypagestyle{plain}{%
  \renewcommand{\headrulewidth}{0pt}%
  \fancyhf{}%
  \fancyfoot[L]{Metodologia científica - UFG}%
  \fancyfoot[R]{\footnotesize Página \thepage\ de \pageref{LastPage}}%
}


%head and foot note
\lhead{\textbf{Mestrado em Ciência da Computação} \\\textbf{Projeto de pesquisa}}
\chead{}
\rhead{\textbf{Metodologia Científica}\\ \textit{Airton Bordin Junior}}
\lfoot{}
\cfoot{}
\rfoot{\footnotesize Página \thepage\ de \pageref{LastPage}}

\renewcommand{\headrulewidth}{0.1pt}
\renewcommand{\footrulewidth}{0.1pt}
\makeindex
\begin{document}

\begin{titlepage}

\newcommand{\HRule}{\rule{\linewidth}{0.5mm}} % Defines a new command for the horizontal lines, change thickness here

\begin{center} % Center everything on the page
 
%----------------------------------------------------------------------------------------
%	HEADING SECTIONS
%----------------------------------------------------------------------------------------

\begin{figure}
\centering
\begin{minipage}{.5\textwidth}
  \centering
  \includegraphics[width=.4\linewidth,scale=1.3]{logo-uel}
\end{minipage}%
\begin{minipage}{.5\textwidth}
  \centering
  \includegraphics[scale=0.2]{logo-ufg}
%    \includegraphics[width=.4\linewidth,scale=0.1]{logo-ufg}

\end{minipage}
\end{figure}

%\includegraphics[scale=0.5]{logo-ufg}\\[1.0cm] % Include a department/university logo - this will require the graphicx package

%\textsc{\LARGE Universidade Federal de Goi�s}\\[1.5cm] % Name of your university/college

%\textsc{\Large \textbf{Instituto de Computa��o}}\\[0.5cm] % Major heading such as course name
\textsc{\large \textbf{Mestrado em Ciência da Computação}}\\[0.5cm]
\textsc{\large \textbf{Disciplina de Metodologia Científica}}\\[1.5cm] % Minor heading such as course title

%----------------------------------------------------------------------------------------
%	TITLE SECTION
%----------------------------------------------------------------------------------------

\HRule \\[0.4cm]
{ \huge \bfseries Expansão automatizada de Léxicos para a Análise de Sentimentos por meio de programação evolucionária}\\[0.4cm] % Title of your document
\HRule \\[1.5cm]

%----------------------------------------------------------------------------------------
%	AUTHOR SECTION
%----------------------------------------------------------------------------------------

\begin{framed}
\emph{Autor:} Airton Bordin Junior
\par
\emph{Orientador:} Nádia Félix Felipe da Silva
\par
\emph{Coorientador:} Celso Gonçalves Camilo Junior
\end{framed}

\vspace*{3cm}

%----------------------------------------------------------------------------------------
%	DATE SECTION
%----------------------------------------------------------------------------------------

{\large \today}\\ % Date, change the \today to a set date if you want to be precise

\end{center}

%----------------------------------------------------------------------------------------

\vfill % Fill the rest of the page with whitespace

\end{titlepage}

\section{Apresentação}
Airton Bordin Junior, bacharel em Ciência da Computação. Cursou os primeiros 3 anos do curso na Universidade Estadual do Oeste do Paraná (UNIOESTE) campus Foz do Iguaçu, e finalizou a graduação na faculdade Anglo Americano, na mesma cidade, no ano de 2011.

Possui, também, graduação em Gestão Pública, pelo Instituto Federal de Santa Catarina, cursado por meio da UaB, no campus de Foz do Iguaçu.

Após a graduação, cursou especialização em Redes de Computadores pela Universidade Federal Tecnológica do Paraná, campus de Cornélio Procópio e MBA em Gerenciamento de Projetos pelo Centro Universitário Dinâmica das Cataratas, em Foz do Iguaçu.

Atua profissionalmente na área da computação desde 2010. Trabalhou como desenvolvedor de software, testador e, por fim, como Analista de Sistemas no Parque Tecnológico Itaipu, responsável pela área de TI do projeto de Segurança de Barragens

Lecionou 2 semestres no curso Técnico em Informática Para Internet do Pronatec, atuando nas disciplinas de Sistemas Operacionais e Segurança de Sistemas, e 1 semestre no curso de Ciência da Computação em uma faculdade local, lecionando as disciplinas de Processamento de Imagens e Sistemas Inteligentes.

Sempre quis continuar estudando, e o mestrado era um objetivo a ser atingido. Por conta da falta de oportunidades na cidade onde morava (Foz do Iguaçu), esta meta teve que ser adiada. Hoje, tem a oportunidade e a honra de participar como aluno regular do programa de mestrado pela Universidade Federal de Goiás, na linha de pesquisa de Inteligência Computacional.

Prentende aprofundar o trabalho na área de Mineração de Opiniões, também chamada de Análise de Sentimentos, mais precisamente na criação e expansão automatizada de dicionários léxicos, alinhado com trabalhos em andamento de alguns professores da Universidade. O interesse em pesquisar esse assunto vem crescendo nos últimos anos, principalmente com o aumento da produção de conteúdo na WEB, e apresenta-se como um desafio interessante e atual e que pode trazer benefícios para diversas outras áreas de conhecimento como, por exemplo, o setor de saúde.

\begin{itemize}
\item{Principais razoes que o candidato e um otimo nome para desenvolver o tema escolhido.}
\end{itemize}

\section{Resumo}
Deve ter aproximadamente 300 palavras. Alem disto, deve ter uma descricao breve de todo o projeto, atendendo a estas quatro areas:

\begin{itemize}
\item{O que voce vai fazer? (o problema)}
\item{Como sera feito (metodologia)}
\item{Resultados esperados? (apenas os mais relevantes)}
\item{Qual a importancia destes? (Conclusoes / recomendacoes)}
\end{itemize}

\section{Introdução}

A Mineração de Opiniões, também chamada de Análise de Opiniões ou Análise de Sentimentos, é uma linha de pesquisa abrangente e que vem sendo tema de diversos trabalhos nos últimos anos, principalmente devido ao aumento no número de usuários de Internet e o consequente crescimento da produção de conteúdo independente na rede, como opiniões, avaliações, entre outros. 

Essa área de estudo tem como principal desafio a Análise de Opiniões, descritas em linguagem natural, para a identificação da polaridade implícita ou explícita no texto. Essa polaridade é, na maior parte das vezes, identificada como uma escala de pontuação de sua característica positiva, negativa ou neutra.

Uma das principais técnicas utilizadas para a Análise de Sentimentos é a realizada por meio de dicionários de dados. Esses dicionários contêm palavras previamente avaliadas por especialistas humanos, principalmente quanto à sua polaridade. Neste contexto, esse conjunto de palavras, juntamente com suas polaridades, é chamado de Dicionário Léxico. Algumas bases, como a \emph{WordNet} - utilizada na maior parte dos trabalhos - apresenta, também, os sinônimos para cada palavra.

Porém, é evidente a limitação inerente à estratégia de utilização do Dicionário Léxico - a própria lista de palavras disponíveis. Esse fato muitas vezes limita a realização de uma análise mais profunda sobre determinado contexto. Nesse sentido, um dos principais desafios na área de Mineração de Opiniões é a criação e ampliação do Dicionário Léxico de forma automatizada, tema central do presente trabalho. Grande parte desses dicionários são construídos de forma manual, fato que caracteriza uma limitação óbvia para a maior parte dos contextos e domínios. 

Consciente dessa limitação, a ideia principal do presente trabalho é a criação de um processo automatizado de expansão de Dicionário Léxico, sensível a domínios específicos, fazendo uso de técnicas de algoritmos bioinspirados, mais precisamente Programação Evolucionária. Resumidamente, o processo fará a adequação das polaridades das palavras contidas no dicionário de forma a maximizar a corretude das avaliações. Para avaliar a taxa de erro de cada solução parcial gerada, serão utilizados conjuntos conhecidos de documentos previamente avaliados por especialistas humanos, de forma a compará-los com a solução gerada pelo sistema. Para a avaliação, sistemas de classificação de sentimentos disponíveis serão utilizados.

Devido à característica intrínseca do próprio problema, serão utilizadas na pesquisa bases de textos em inglês. Ao mesmo tempo, muitas técnicas utilizadas durante o trabalho também poderão ser utilizadas para resolver problemas em português, com as devidas alterações. 

Espera-se que esse processo, bem como os Dicionários Léxicos por ele gerado, possam ser utilizados como entrada de processos de avaliação em diversas áreas de Mineração de Opiniões como, por exemplo, Análise de Sentimentos em redes sociais. Algumas dessas pesquisas são realizadas na própria instituição, apoiando, assim, o trabalho de outros pesquisadores. Além disso, devido ao caráter automatizado dessa solução proposta, o mesmo processo poderá ser utilizado, avaliado e melhorado para outras situações, contextos e idiomas.

Por fim, a pesquisa e a utilização de diversas técnicas de PNL e expansão automatizada de Dicionário Léxico poderão servir como um \emph{benchmark} dos principais métodos e classificadores, auxiliando na escolha de ferramentas e abordagens para trabalhos futuros em contextos específicos.

Para apoiar a clareza e desenvolvimento da proposta de trabalho, o presente documento está estruturado da seguinte forma: o próximo capítulo tratará da descrição do problema, apresentando as principais limitações e dificuldades no contexto de Análise de Sentimentos e Dicionários Léxicos. O capítulo \ref{sec:obj} tratará dos objetivos gerais e específicos. A revisão bibliográfica, apresentada no capítulo \ref{sec:bibl}, tem por objetivo o embasamento teórico para apoiar nas soluções propostas, apresentando o estado da arte sobre o assunto, bem como definindo conceitos fundamentais para entender e trabalhar com a Análise de Sentimentos. O capítulo \ref{sec:impact} apresenta o impacto científico da solução e suas possíveis contribuções para a área. A metodologia, descrita no capítulo \ref{sec:met}, descreve a forma como serão desenvolvidas cada uma das etapas do processo, seguida do capítulo com uma previsão de cronograma do trabalho. Resultados esperados são descritos no capítulo \ref{sec:result}, seguidos da identificação dos colaboradores e participantes do projeto e, por fim, as referências bibliográficas utilizadas na proposta.


\section{Descrição do Problema}
Nesta secao, o proponente tem a oportunidade de discorrer livremente sobre o problema a ser estudado, de forma a descrever e justificar o problema aos possiveis leitores.  A revisao bibliografica, identificacao de projetos semelhantes e possiveis "brechas" em trabalhos ja realizados tambem sao identificados aqui, de forma a justificar, atraves de periodos e outros aceitos pela comunidade cientifica, a existencia do problema.

\section{Objetivos}
\label{sec:obj}
\subsection{Objetivo geral}
O presente trabalho tem por objetivo a criação de um sistema para a expansão de um dicionário léxico, contendo palavras e suas respectivas orientações semânticas (positiva, negativa, neutra) para um determinado contexto. A criação de um léxico abrangente e específico para um determinado contexto é um desafio para a área de Análise de Sentimentos, e fundamental para o correto funcionamento de todo o processo de análise de dados.

\subsection{Objetivos especificos}

\begin{enumerate}
\item Criação de um sistema para a expansão automatizada de dicionário léxico para domínios específicos;
\item Criação de dicionários léxicos consolidados para domínios específicos, prontos para serem utilizados por sistemas de Análise de Sentimentos;
\item Estudo comparativo da técnica proposta com outras técnicas da literatura, de forma a apoiar a evolução de soluções existentes;
\item Publicação de trabalhos sobre o assunto de forma a expandir o conhecimento sobre a utilização de algoritmos bioinspirados na área de Análise de Sentimentos.

\end{enumerate}

\section{Revisão bibliográfica}
\label{sec:bibl}
Nesta secao, que deve ocupar \textbf{no maximo} 3 paginas, devera ser apresentada a revisao bibliografica. O numero minimo de artigos e de 20 vinte (internacionais) e 10 (dez) nacionais.

\section{Impacto Científico}
\label{sec:impact}
A técnica de Análise de Sentimento por meio de um dicionário léxico é uma das mais utilizadas na literatura. Mostra-se, portanto, essencial a obtenção de um conjunto de palavras consolidado, juntamente com as orientações semânticas respectivas. 
Uma palavra pode ter um significado e, consequentemente, uma polaridade diferente, dependendo do contexto ao qual está inserido. 
Um conjunto léxico inadequado leva a análises inconsistentes, prejudicando o resultado final do sistema.
A solução proposta neste trabalho criará, de forma automatizada, léxicos para diferentes domínios, que poderão servir como entrada para diversos classificadores e sistemsas de análise de sentimentos. Além disso, a técnica pode ser utilizada em outros idiomas, de forma a suprir uma carência de dicionários consistentes em linguagens pouco conhecidas.
O conjunto de palavras gerado pela solução proposta poderá ser utilizado como \emph{benchmark} para outros trabalhos na área, bem como ser expandido com outras técnicas adequadas.

\section{Metodologia}
\label{sec:met}
A metodologia descreve a forma como serao desenvolvidas cada uma das etapas do processo. Estas devem ser as mesmas que foram apresentadas no cronograma de trabalho, descrito na secao \ref{sec:crono}.

\subsection{Informacoes importantes}
\begin{itemize}
\item{Pergunta a ser respondida: Como voce atingira os objetivos da pesquisa?}
\item{Divida o seu projeto de pesquisa em diversas etapas.}
\item{Cada etapa devera ter uma breve descricao informando ao leitor \textbf{como} a mesma sera executada.}
\item{Faca uma previsao de tempo de cada uma das etapas.}
\item{Ao final de cada etapa, apresente um \textbf{marco fisico}\footnote{Relatorio, artigo, projeto de sistema, etc.}}
\end{itemize}

\subsection{Sugestao}
\begin{enumerate}[D1.]
\item{\textbf{Revisao bibliografica:} Nesta etapa do trabalho sera feita uma revisao bibliografica com vistas a identificar o estado da arte do problema que esta sendo proposto. Importante registrar que esta revisao bibliografica seguira os moldes propostos por \cite{Kitchenham2004}}. Serao consultadas as bases de dados do [COLOCAR BASES AQUI] \emph{Portal da Capes}, \emph{IEEEXplore} e \emph{ACM Digital Library}.

\item{\textbf{Estudo dos principais classificadores de sentimentos:}}
Nesta etapa será feita uma pesquisa sobre os principais classificadores disponíveis, preferencialmente livres e \emph{open source}, utilizados para a análise de sentimentos, e que permitem a manipulação de seu dicionário. O objetivo principal desse passo é selecionar ferramentas que proporcionarão dados comparativos para teste da solução proposta.

\item{\textbf{Análise dos principais léxicos disponíveis:}}
Busca pelos principais conjuntos de palavras, e suas respectivas polaridades, disponíveis para utilização. Esses léxicos servirão como base e material de testes para a solução.

\item{\textbf{Recuperação das principais bases de opiniões anotadas disponíveis:}}
Busca e recuperação de bases de opiniões anotadas e consistentes, representando resultados confiáveis e corretos. Essas bases, já revisadas por especialistas humanos, servirão como parâmetro de corretude da solução proposta, bem como serão utilizadas para o cálculo de erro dos resultados obtidos.

\item{\textbf{Implementação da solução:}}
Implementação da solução proposta, com o objetivo da expansão de um léxico, sensível a um domínio específico, que servirá como entrada para um classificador utilizado em processos de análise de sentimentos.

\item{\textbf{Teste da solução com os classificadores selecionados:}}
Teste dos resultados fazendo uso dos classificadores selecionados anteriormente, de forma a obter um resultado satisfatório, minimizando a taxa de erros ao comparar com resultados consolidados e previamente revisados.

\item{\textbf{Levantamento dos dados de testes e relatórios:}}
Levantamento dos dados da utilização da solução, fazendo uso dos classificadores selecionados, e fazendo a comparação com outros sistemas e soluções disponíveis na literatura. Essa etapa fará a classificação e organização dos resultados, de forma a facilitar a visualização, entendimento, e auxiliar na tomada de decisões sobre o projeto.
\end{enumerate}

\subsection{Marcos fisicos}
\begin{enumerate}[D1.]
\item{Documento com a revisao bibliografica.}
\item{Lista dos principais classificadores.}
\item{Lista dos principais léxicos.}
\item{Bases de opiniões recuperadas.}
\end{enumerate}

\section{Cronograma de trabalho}
\label{sec:crono}

\begin{ganttchart}{1}{24}
\gantttitle{2017}{10}\gantttitle{2018}{12}\gantttitle{2019}{2} \\
\gantttitlelist{3,...,12}{1}\gantttitlelist{1,...,12}{1}\gantttitlelist{1,...,2}{1} \\
\ganttgroup{Disciplinas}{1}{9} \\
\ganttmilestone{Fim do $1^o$ semestre}{5} \ganttnewline
\ganttmilestone{Fim do $2^o$ semestre}{10} \ganttnewline
\ganttgroup{Dissertacao}{1}{23} \\
\ganttbar{$D_1$}{1}{4} \\
\ganttbar{$D_2$}{3}{4} \\
\ganttbar{$D_3$}{4}{6} \\
\ganttbar{$D_4$}{5}{7} \\
\ganttbar{$D_5$}{6}{14} \\
\ganttbar{$D_6$}{12}{16} \\
\ganttbar{$D_7$}{17}{20} \\
%\ganttlink{elem5}{elem6}
\ganttmilestone{Relatorio 1}{4} \ganttnewline
\ganttmilestone{Relatorio 2}{9} \ganttnewline
\ganttmilestone{Relatorio 3}{14} \ganttnewline
\ganttlink{elem9}{elem10}
\ganttmilestone{Qualificacao}{17}\ganttnewline
\ganttmilestone{Defesa}{19}
\end{ganttchart}

\begin{center}
\large \textbf{Legenda}
\end{center}

\begin{enumerate}[D1.]
\item{Revisao bibliografica.}
\item{Estudo dos principais classificadores de sentimentos.}
\item{Análise dos principais léxicos disponíveis.}
\item{Recuperação das principais bases de opiniões anotadas disponíveis.}
\item{Implementação da solução.}
\item{Teste da solução com os classificadores selecionados.}
\item{Levantamento dos dados de testes e relatórios.}
\end{enumerate}

\section{Resultados Esperados}
\label{sec:result}
Espera-se, com o presente trabalho, a criação de um processo automatizado de expansão de léxico dependente de domínio, fazendo uso de técnicas de algoritmos evolucionários. Nesse sentido, expansão significa tanto a criação e definição da orientação semântica de novas palavas, bem como a alteração das polaridades das palavras já existentes para um valor mais adequado ao domínio que trata o processo.
Pela característica genérica da solução, a criação de diversos léxicos para vários domínios diferentes é limitada tão somente à escolha dos contextos específicos e à disponibilidade de dados anotados para teste da solução.
Podemos citar, também, uma possível melhoria em algumas técnicas de Análise de Sentimentos que fazem uso de léxicos padrão, contribuindo assim para a evolução de outros sistemas de Mineração de Opiniões que usam a estratégia de dicionário.
Os resultados parciais e finais do trabalho serão descritos em artigos científicos que serão submetidos à eventos na área, de forma a compartilhar o conhecimento e avanços alcançados pela técnica proposta.

\subsection{Algoritmos} % sugestao 
Será desenvolvido um algoritmo que criará e/ou ampliará, de forma automatizada, um léxico para um domínio específico que será utilizado como entrada para um sistema classificador de Análise de Sentimentos.
Esse software fará uso de técnicas de algoritmos bioinspirados, mais precisamente Programação Evolucionária, para a atribuição de valores sentimentais para cada palavra, de forma a maximizar a taxa de acerto ao ser processado por um classificador existente.
Ao passo que o algoritmo é independente de domínio, pode ser utilizado, desde que haja dados de testes suficientes, para qualquer contexto desejado.

\subsection{Artigos cientificos}
\begin{tabular}{| c | c | c | c |}
\hline
\textbf{Quantidade} & \textbf{Qualis} & \textbf{Tipo} & \textbf{Nome} \\
\hline
1 & B1 & Conferencia & FIE - Frontiers in Education \\
\hline
1 & A1 & Periodico & Computers in Education \\
\hline
\end{tabular}

\section{Identificacao dos Participantes e Colaboradores}
Aqui o candidato devera descrever se o projeto que esta sendo proposto faz parte de um projeto de pesquisa maior ou nao. Alem disto, deve descrever as possiveis colaboracoes (alunos de iniciacao cientifica, mestrado ou doutorado) que possam contribuir para o seu trabalho. 

\section{Referencias bibliograficas}
\bibliographystyle{apalike}
\bibliography{projeto}
\end{document}