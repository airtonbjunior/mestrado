\documentclass[a4paper,11pt]{article}
\usepackage[top=2cm, bottom=2.5cm, left=2cm, right=2cm]{geometry}

\usepackage[utf8]{inputenc}
\usepackage[brazil]{babel}
\usepackage{color}
\usepackage{amsmath}
\usepackage{amsfonts}
\usepackage{pdfpages}
\usepackage{lastpage}
\usepackage{framed}
\usepackage{tgbonum}
\usepackage{enumitem}

%Manipulação de cabeçalho e rodapé
\usepackage{fancyhdr}
%Distância do cabeçalho em relação ao topo do página
\setlength{\headheight}{45.0pt}
%Largura das linhas de cabeçalho e rodapé
\renewcommand{\footrulewidth}{0.4pt}
\renewcommand{\headrulewidth}{0.4pt}
\setlength{\headwidth}{\textwidth}
%Inclui o logo da UEL no lado esquerdo do cabeçalho
\fancyhead[L]{
 \includegraphics[height=0.53in]{p-modelo1.jpg}
}
%Inclui o logo do DC no lado direito do cabeçalho
\fancyhead[R]{
   \includegraphics[height=0.53in]{logo_ufg.png}
}
\pagestyle{fancy}
\cfoot{}
\rfoot{\textbf{\footnotesize Página \thepage\ de \pageref{LastPage}}}
\lfoot{\textbf{IPC/MET - 2017/1 - Problema de pesquisa}}
%opening

\begin{document}
\fontfamily{cmss}\selectfont

% \maketitle

% \begin{abstract}

% \end{abstract}
\begin{framed}
\begin{center}
\textbf{\Large{Problema de pesquisa}}

\textbf{Aluno: INCLUIR O SEU NOME COMPLETO}
\end{center}

\textbf{Subárea do conhecimento do CNPq:} Colocar a subárea.

\textbf{Subsubárea do conhecimento:} 

\textbf{Linha de pesquisa:} \emph{Alunos da UFG / Imperatriz / Tocantins devem colocar a linha específica do Instituto. O pessoal da UEL, deve verificar a linha específica relacionada ao docente.}

\end{framed}

\section{Contextualização}
Aqui, o aluno deverá fazer um detalhamento de \textbf{pelo menos} 10 (dez) artigos científicos que apresentem resultados significativos para a área em que está propondo o problema. Os artigos deverão estar devidamente citados no texto. Deve-se ter uma atenção especial na escrita do texto, de modo que este \emph{flua}. Na prática, significa que os parágrafos devem ter conexões entre si. 

Um outro ponto a ser lembrado, diz respeito ao processo de busca dos artigos na internet. Em um primeiro momento, e a partir da proposta a ser descrita, devem ser definidas as palavras chave de busca. Estas tem como função apontar para os sites de pesquisa (que também devem ser definidos), quais artigos estão sendo procurados.  

A próxima etapa é a leitura dos abstracts. Estes devem apontar para o leitor uma apresentação completa do artigo, sem entrar em detalhes. A partir desta leitura é que devem ser feitos os downloads e posteriormente a leitura dos artigos. Na sequência, deve então ser escrita a contextualização do problema. Caso o aluno não tenha experiência, sugere-se a leitura do artigo: \cite{kitchenham2009systematic}. O mesmo é um guia sobre como fazer uma revisão sistemática (a próxima fase da disciplina).

A contextualização deverá ocupar \emph{a primeira página deste documento}. Espera-se que a mesma tenha de 4 a 5 parágrafos com 4 a 5 linhas cada um destes. Os excessos serão avaliados de forma negativa.

\section{Problema}
Aqui deverá ser apresentado o problema da dissertação / tese a ser resolvido. Espera-se uma descrição detalhada do problema, que deverá ser uma extensão da contextualização. Também deve ser apontada qual será a contribuição do mesmo para a ciência. 

\bibliographystyle{apalike}
\bibliography{problema}

\end{document}