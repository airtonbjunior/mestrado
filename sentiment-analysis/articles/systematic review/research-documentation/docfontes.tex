\documentclass[a4paper,11pt]{article}
\usepackage[top=2cm, bottom=2.5cm, left=2cm, right=2cm]{geometry}

\usepackage[utf8]{inputenc}
\usepackage[brazil]{babel}
\usepackage{color}
\usepackage{amsmath}
\usepackage{amsfonts}
\usepackage{pdfpages}
\usepackage{lastpage}
\usepackage{framed}
\usepackage{tgbonum}
\usepackage{enumitem}
\usepackage{multirow}

%Manipulação de cabeçalho e rodapé

\usepackage{fancyhdr}

%Distância do cabeçalho em relação ao topo do página
\setlength{\headheight}{45.0pt}
%Largura das linhas de cabeçalho e rodapé
\renewcommand{\footrulewidth}{0.4pt}
\renewcommand{\headrulewidth}{0.4pt}
\setlength{\headwidth}{\textwidth}
%Inclui o logo da UEL no lado esquerdo do cabeçalho
\fancyhead[L]{
 \includegraphics[height=0.53in]{p-modelo1.jpg}
}
\fancyhead[C]{}
\fancyhead[R]{IPC/MET $1^o$ semestre/2017 \\ Prazo: 19/06/2017}

\pagestyle{fancy}
\cfoot{}
\rfoot{\textbf{\footnotesize Página \thepage\ de \pageref{LastPage}}}
\lfoot{\textbf{IPC/MET - 2017/1 - Documentação das fontes de pesquisa}}
%opening

\begin{document}
%\fontfamily{cmss}\selectfont

\begin{framed}
\begin{center}
\textbf{Aluno: Airton Bordin Junior}
\end{center}
\end{framed}

Este documento foi baseado em \cite{kitchenham2007guidelines}. A documentação deverá ser entregue de acordo com as orientações que estão neste documento. Conforme pode ser observado, existem diferentes tabelas que deverão conter as informações do processo de busca, a saber: A tabela \ref{tab:tab_bases}, deverá apresentar a descrição das bases de dados que foram consultadas, os seus respectivos endereços da internet, a data em que a busca foi efetuada e por fim, os anos que foram cobertos pela busca. 

\section{Bases de dados}
\begin{table}[ht]
\centering
\begin{tabular}{| c | c | c | c |}
\hline
\textbf{Base de dados} &  \textbf{Data da busca} & \textbf{Anos cobertos na busca} & \textbf{Idioma} \\
\hline
B1 & dd/mm/aaaa & 9999 até 9999 & \multirow{2}{*}{} Português \\
\cline{4-4}
& & & Inglês \\
\hline

\end{tabular}
\caption{Relação de bases de dados consultadas com respectivas datas de busca}
\label{tab:tab_bases}
\end{table}

Um ponto importante a ser observado na tabela \ref{tab:tab_bases}, diz respeito ao formato em que as informações devem ser apresentadas. A data da consulta deverá ter o formato dd/mm/aaaa. Caso tenham sido feitas consultas em diferentes dias, estes deverão estar indicados. O período coberto pela busca diz respeito ao ano inicial / final em que os trabalhos foram publicados na respectiva base de dados. Exemplo: 1995 até 2017. Também deve estar presente qual foi o idioma utilizado na busca em cada uma das bases de dados. Caso tenham sido consultadas em português e inglês a informação deverá estar presente.

\section{Palavras chave de busca}
\begin{table}[ht]
\centering
\begin{tabular}{| l | l |}
\hline
\textbf{Palavra-chave} & \textbf{Idioma} \\
\hline
\end{tabular}
\caption{Detalhamento das palavras chave da busca e respectivo idioma}
\label{tab:tab_palchaves}
\end{table}

Na tabela \ref{tab:tab_palchaves} devem ser informadas as palavras de busca com o respectivo idioma. 
 
\section{Detalhamento da estratégia de busca}
 Aqui o aluno deverá descrever os procedimentos de busca nas diferentes bases de dados apresentadas na tabela \ref{tab:tab_bases}. Estas informações referem-se  ao tipo de busca efetuado e a aplicação dos critérios de inclusão / exclusão. Caso a busca tenha sido feita em periódicos específicos ou anais de conferências, estas informações deverão estar detalhadas conforme as tabelas \ref{tab:tab_periodico} e  \ref{tab:tab_conf}. Caso a busca tenha sido feita apenas em bases de dados, as informações abaixo deverão ser suprimidas do documento a ser entregue.
 
\begin{table}[ht]
\begin{tabular}{|c | c | c |}
\hline
\textbf{Nome} & \textbf{Intervalo de busca} & \textbf{Informações extras} \\
\hline
Journal of XXXX & 2000 até 2017 & Foram localizados 43 artigos na busca. \\
\hline
\end{tabular}
\caption{Relação de periódicos consultados}
\label{tab:tab_periodico}
\end{table}

Na coluna de \emph{Informações extras} da tabela \ref{tab:tab_periodico}, devem ser registradas as informações à respeito dos resultados da busca. 

\begin{table}[ht]
\begin{tabular}{|c | c | c |}
\hline
\textbf{Nome} & \textbf{Ano do evento} & \textbf{Informações extras} \\
\hline
CBIE & 2016 & Foram localizados 12 artigos na busca. \\
\hline
\end{tabular}
\caption{Relação de anais de conferências consultados}
\label{tab:tab_conf}
\end{table}

\bibliographystyle{apalike}
\bibliography{docfontes}

\end{document}