\documentclass[a4paper,11pt]{article}
\usepackage[top=2cm, bottom=2.5cm, left=2cm, right=2cm]{geometry}

\usepackage[utf8]{inputenc}
\usepackage[brazil]{babel}
\usepackage{color}
\usepackage{amsmath}
\usepackage{amsfonts}
\usepackage{pdfpages}
\usepackage{lastpage}
\usepackage{framed}
\usepackage{tgbonum}
\usepackage{enumitem}
\usepackage{multirow}
\usepackage{makecell}


%Manipulação de cabeçalho e rodapé

\usepackage{fancyhdr}

%Distância do cabeçalho em relação ao topo do página
\setlength{\headheight}{45.0pt}
%Largura das linhas de cabeçalho e rodapé
\renewcommand{\footrulewidth}{0.4pt}
\renewcommand{\headrulewidth}{0.4pt}
\setlength{\headwidth}{\textwidth}
%Inclui o logo da UEL no lado esquerdo do cabeçalho
\fancyhead[L]{
 \includegraphics[height=0.53in]{p-modelo1.jpg}
}
\fancyhead[C]{}
\fancyhead[R]{IPC/MET $1^o$ semestre/2017 \\ Prazo: 19/06/2017}

\pagestyle{fancy}
\cfoot{}
\rfoot{\textbf{\footnotesize Página \thepage\ de \pageref{LastPage}}}
\lfoot{\textbf{IPC/MET - 2017/1 - Documentação das fontes de pesquisa}}
%opening

\begin{document}
%\fontfamily{cmss}\selectfont

\begin{framed}
\begin{center}
\textbf{Aluno: Airton Bordin Junior}
\end{center}
\end{framed}

\section{Bases de dados}
\begin{table}[ht]
\centering
\begin{tabular}{| c | c | c | c |}
\hline
\textbf{Base de dados} &  \textbf{Data da busca} & \textbf{Anos cobertos na busca} & \textbf{Idioma} \\
\hline
\makecell{\emph{Science Direct} \\ http://www.sciencedirect.com/} & 22/05/2017 & 2000 até 2017 & Inglês \\
\hline
\makecell{\emph{IEEEXplore} \\ http://ieeexplore.ieee.org/} & 24/05/2017 & 2000 até 2017 & Inglês \\
\hline
\makecell{\emph{ACM Digital Library} \\http://dl.acm.org/} & 24/05/2017 & 2000 até 2017 & Inglês \\
\hline
\makecell{\emph{Research Gate} \\ https://www.researchgate.net/} & 29/05/2017 & 2000 até 2017 & Inglês \\
\hline
\makecell{\emph{Semantic Scholar} \\ https://www.semanticscholar.org/} & 01/06/2017 & 2000 até 2017 & Inglês \\
\hline

\end{tabular}
\caption{Relação de bases de dados consultadas com respectivas datas de busca}
\label{tab:tab_bases}
\end{table}

\section{Palavras chave de busca}
\begin{table}[h]
\centering
\begin{tabular}{| c | c |}
\hline
\textbf{Palavra-chave} & \textbf{Idioma} \\
\hline
\makecell{pub-date \textgreater 1999 and (Sentiment Analysis OR Opinion Mining) AND \\ (Lexicon Expansion) AND (Genetic Programming OR Genetic Algorithm) \\ AND (Semantic Orientation)[All Sources(Computer Science)]} & Inglês \\
\hline
\makecell{pub-date \textgreater 1999 and (Sentiment Analysis OR Opinion Mining) AND \\ (Lexicon Expansion OR Lexicon) \\ AND (Semantic Orientation)[All Sources(Computer Science)]} & Inglês \\
\hline
\makecell{((Sentiment Analysis OR Opinion Mining) AND (Lexicon Expansion OR Lexicon) \\ AND (Semantic Orientation))  and refined by Year: 2000-2017} & Inglês \\
\hline
\makecell{"query": { (+Sentiment +Analysis Lexicon Expansion Genetic Algorithm) } \\ "filter": {"publicationYear":{ "gte":2000 }},
{owners.owner=HOSTED}} & Inglês \\
\hline
\end{tabular}
\caption{Detalhamento das palavras chave da busca e respectivo idioma}
\label{tab:tab_palchaves}
\end{table}
 
\section{Detalhamento da estratégia de busca}
 Aqui o aluno deverá descrever os procedimentos de busca nas diferentes bases de dados apresentadas na tabela \ref{tab:tab_bases}. Estas informações referem-se  ao tipo de busca efetuado e a aplicação dos critérios de inclusão / exclusão. Caso a busca tenha sido feita em periódicos específicos ou anais de conferências, estas informações deverão estar detalhadas conforme as tabelas \ref{tab:tab_periodico} e  \ref{tab:tab_conf}. Caso a busca tenha sido feita apenas em bases de dados, as informações abaixo deverão ser suprimidas do documento a ser entregue.
 
\begin{table}[ht]
\begin{tabular}{|c | c | c |}
\hline
\textbf{Nome} & \textbf{Intervalo de busca} & \textbf{Informações extras} \\
\hline
Journal of XXXX & 2000 até 2017 & Foram localizados 43 artigos na busca. \\
\hline
\end{tabular}
\caption{Relação de periódicos consultados}
\label{tab:tab_periodico}
\end{table}

Na coluna de \emph{Informações extras} da tabela \ref{tab:tab_periodico}, devem ser registradas as informações à respeito dos resultados da busca. 

\begin{table}[ht]
\begin{tabular}{|c | c | c |}
\hline
\textbf{Nome} & \textbf{Ano do evento} & \textbf{Informações extras} \\
\hline
CBIE & 2016 & Foram localizados 12 artigos na busca. \\
\hline
\end{tabular}
\caption{Relação de anais de conferências consultados}
\label{tab:tab_conf}
\end{table}

\bibliographystyle{apalike}
\bibliography{docfontes}

\end{document}