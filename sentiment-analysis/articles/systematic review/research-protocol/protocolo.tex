\documentclass[a4paper,11pt]{article}
\usepackage[top=2cm, bottom=2.5cm, left=2cm, right=2cm]{geometry}

\usepackage[utf8]{inputenc}
\usepackage[brazil]{babel}
\usepackage{color}
\usepackage{amsmath}
\usepackage{amsfonts}
\usepackage{pdfpages}
\usepackage{lastpage}
\usepackage{framed}
\usepackage{tgbonum}
\usepackage{enumitem}

%Manipulação de cabeçalho e rodapé
\usepackage{fancyhdr}
%Distância do cabeçalho em relação ao topo do página
\setlength{\headheight}{45.0pt}
%Largura das linhas de cabeçalho e rodapé
\renewcommand{\footrulewidth}{0.4pt}
\renewcommand{\headrulewidth}{0.4pt}
\setlength{\headwidth}{\textwidth}
%Inclui o logo da UEL no lado esquerdo do cabeçalho
\fancyhead[L]{
 \includegraphics[height=0.53in]{logo-ufg.png}
}
\fancyhead[C]{}
\fancyhead[R]{IPC/MET $1^o$ semestre/2017 \\ Prazo: 19/06/2017}
%Inclui o logo do DC no lado direito do cabeçalho
%\fancyhead[R]{
%   \includegraphics[height=0.53in]{indice.jpeg}
%}
\pagestyle{fancy}
\cfoot{}
\rfoot{\textbf{\footnotesize Página \thepage\ de \pageref{LastPage}}}
\lfoot{\textbf{IPC/MET - 2017/1 - Protocolo da Revisão Sistemática}}
%opening

\begin{document}
\fontfamily{cmss}\selectfont

% \maketitle

% \begin{abstract}

% \end{abstract}
\begin{framed}
\begin{center}
\textbf{Aluno: Airton Bordin Junior}
\end{center}

\center{Revisão Sistemática sobre técnicas de criação e expansão automatizada de Dicionários Léxicos no contexto de Análise de Sentimentos}

\end{framed}

\section{Justificativa}
A Revisão Sistemática da Literatura fornece uma forma estruturada, objetiva e reprodutível de identificar, avaliar e interpretar trabalhos relevantes em uma determinada área de conhecimento. A análise desses trabalhos apoiará a resoluções das questões de pesquisa, que devem ser respondidas pelo presente projeto.

Ao documentar a estratégia e planejamento da revisão, a pesquisa torna-se mais confiável e imparcial, dificultando a seleção somente de artigos que corroboram à tese do pesquisador. 

Portanto, de forma a identificar os trabalhos primários e o estado da arte sobre o tema do presente projeto, bem como a seleção e avaliação imparcial dos artigos, justifica-se a utilização do processo de Revisão Sistemática. Além disso, será possível responder as perguntas da pesquisa e identificar \emph{gaps} na área de estudo.

\section{Perguntas}
Aqui deverão ser apresentadas as perguntas a serem respondidas a partir da revisão sistemática.

\section{Palavras chave}
\begin{itemize}
	\item{Sentiment Analysis}
	\item{Opinion Mining}
	\item{Lexicon Expansion}
	\item{Genetic Algorithms}
	\item{Genetic Programming}
	\item{Semantic Orientation}
\end{itemize}

\section{Checklist}\footnote{Adaptado de \cite{kitchenham2007guidelines}}
\begin{enumerate}
\item{Quais são os objetivos da revisão?}

Encontrar, nas principais bases de dados do tema, os trabalhos relacionados, de forma a identificar o estado da arte e as abordagens mais comuns no contexto do presente projeto.

\item{Quais fontes de pesquisa foram utilizadas para identificar os estudos primários?}

Para a identificação dos estudos primários, buscas serão realizadas nas bases da \emph{Science Direct}, \emph{IEEEXplore}, \emph{ACM Digital Library}, \emph{Research Gate}, \emph{Semantic Scholar}

\item{As fontes utilizadas apresentam algum tipo de restrição?  (S/N)\footnote{Em caso afirmativo, descreva a restrição}}
\item{Quais são os critérios de inclusão e exclusão e como estes serão aplicados?}

Para a presente pesquisa, serão levados em conta artigos publicados após 2010.

\item{Quais foram os critérios utilizados para assegurar a qualidade dos estudos primários?}

\item{Como os critérios de qualidade serão aplicados?}

\item{Como foram extraídos os dados dos estudos primários?}

\item{Como os dados serão sintetizados?}

\item{Como os dados serão combinados?}

\end{enumerate}
\bibliographystyle{apalike}
\bibliography{protocolo}

\end{document}