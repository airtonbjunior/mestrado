\documentclass[a4paper,11pt]{article}
\usepackage[top=2cm, bottom=2.5cm, left=2cm, right=2cm]{geometry}

\usepackage[utf8]{inputenc}
\usepackage[brazil]{babel}
\usepackage{color}
\usepackage{amsmath}
\usepackage{amsfonts}
\usepackage{pdfpages}
\usepackage{lastpage}
\usepackage{framed}
\usepackage{tgbonum}
\usepackage{enumitem}

%Manipulação de cabeçalho e rodapé
\usepackage{fancyhdr}
%Distância do cabeçalho em relação ao topo do página
\setlength{\headheight}{45.0pt}
%Largura das linhas de cabeçalho e rodapé
\renewcommand{\footrulewidth}{0.4pt}
\renewcommand{\headrulewidth}{0.4pt}
\setlength{\headwidth}{\textwidth}
%Inclui o logo da UEL no lado esquerdo do cabeçalho
\fancyhead[L]{
 \includegraphics[height=0.53in]{logo-ufg.png}
}
\fancyhead[C]{}
\fancyhead[R]{IPC/MET $1^o$ semestre/2017 \\ Prazo: 19/06/2017}
%Inclui o logo do DC no lado direito do cabeçalho
%\fancyhead[R]{
%   \includegraphics[height=0.53in]{indice.jpeg}
%}
\pagestyle{fancy}
\cfoot{}
\rfoot{\textbf{\footnotesize Página \thepage\ de \pageref{LastPage}}}
\lfoot{\textbf{IPC/MET - 2017/1 - Protocolo da Revisão Sistemática}}
%opening

\begin{document}
\fontfamily{cmss}\selectfont

% \maketitle

% \begin{abstract}

% \end{abstract}
\begin{framed}
\begin{center}
\textbf{Aluno: Airton Bordin Junior}
\end{center}

\center{Revisão Sistemática sobre técnicas de criação e expansão automatizada de Dicionários Léxicos no contexto de Análise de Sentimentos}

\end{framed}

\section{Justificativa}
A Revisão Sistemática da Literatura fornece uma forma estruturada, objetiva e reprodutível de identificar, avaliar e interpretar trabalhos relevantes em uma determinada área de conhecimento. A análise desses trabalhos apoiará a resoluções das questões de pesquisa, que devem ser respondidas pelo presente projeto.

Ao documentar a estratégia e planejamento da revisão, a pesquisa torna-se mais confiável e imparcial, dificultando a seleção somente de artigos que corroboram à tese do pesquisador. 

Portanto, de forma a identificar os trabalhos primários e o estado da arte sobre o tema do presente projeto, bem como a seleção e avaliação imparcial dos artigos, justifica-se a utilização do processo de Revisão Sistemática. Além disso, será possível responder as perguntas da pesquisa e identificar \emph{gaps} na área de estudo.

\section{Perguntas}
\begin{itemize}
	\item{Quais as principais técnicas empregadas na criação e expansão de Dicionários Léxicos?}
	\item{Estratégias Evolutivas estão sendo utilizadas no contexto de criação e expansão de Dicionários Léxicos?}
	\item{Como a diferença de contexto e domínio vem sendo tratada no contexto de Análise de Sentimentos?}
	\item{Que métricas estão sendo utilizadas para avaliar a qualidade dos algoritmos de criação e expansão de Dicionários Léxicos?}
\end{itemize}

\section{Palavras chave}
\begin{itemize}
	\item{Sentiment Analysis}
	\item{Opinion Mining}
	\item{Lexicon Expansion}
	\item{Genetic Algorithms}
	\item{Genetic Programming}
	\item{Semantic Orientation}
\end{itemize}

\section{Checklist}
\begin{enumerate}
\item{Quais são os objetivos da revisão?}

Encontrar, nas principais bases de dados do tema, os trabalhos primários relacionados, de forma a identificar o estado da arte e as abordagens mais comuns no contexto da presente pesquisa.

\item{Quais fontes de pesquisa foram utilizadas para identificar os estudos primários?}

Para a identificação dos estudos primários, buscas serão realizadas nas bases da \emph{Science Direct}, \emph{IEEEXplore}, \emph{ACM Digital Library}, \emph{Research Gate}, \emph{Semantic Scholar}

\item{As fontes utilizadas apresentam algum tipo de restrição?}

Alguns artigos não estão disponíveis de forma integral e gratuita. 
\item{Quais são os critérios de inclusão e exclusão e como estes serão aplicados?}

Serão realizadas buscas utilizando as palavras-chave, de forma combinada, nas bases supracitadas, de forma a encontrar artigos primários relevantes sobre o tema. Serão selecionados os artigos publicados à partir do ano 2000 e que estejam disponíveis na íntegra para acesso. Caso a base de dados permita, será feita uma ordenação por relevância, baseada na quantidade de citações e importância do trabalho. Além disso, serão considerados somente artigos escritos em inglês.

Feito isso, será feita a leitura dos títulos e, posteriormente, dos resumos de cada trabalho, de forma a aceitar ou rejeitar o trabalho para a próxima fase da Revisão Sistemática. Os trabalhos rejeitados não serão levados em consideração para o desenvolvimento do projeto, mas ainda assim serão utilizados como dados estatísticos.

Para auxiliar nesse processo, será utilizada a ferramenta StArt  \emph{(State of the Art through Systematic Reviews)}, desenvolvida pela Universidade Federal de São Carlos, que fornece funcionalidades para apoiar o pesquisador em todas as fases da Revisão Sistemática da Literatura.

\item{Quais foram os critérios utilizados para assegurar a qualidade dos estudos primários?}

O principal critério utilizado para assegurar a qualidade dos estudos primários é a quantidade de citações que o mesmo possui. Via de regra, um trabalho com muitas citações na literatura pode ser considerado um trabalho de qualidade, uma vez que é utilizado como referência por vários outros. 

Outro critério utilizado é uma avaliação quanto à aderência do trabalho às questões de pesquisa definidos. Uma escala de 0 a 5 foi definida, sendo 0 considerado não aderente e 5 completamente aderente. Como há mais de uma questão de pesquisa definida, foi realizada uma média aritmética das escalas.


\item{Como os critérios de qualidade serão aplicados?}

O critério relacionado à quantidade de citações será feito por meio de ferramentas disponibilizadas pelas próprias bases de dados, que permitem a ordenação dos artigos por quantidade de citações.

A avaliação da escala de aderência do trabalho é feita por meio da leitura do resumo. Caso o mesmo não seja conclusivo o suficiente para a atribuição da nota, é feita a leitura da introdução e conclusão.

Importante salientar que esta análise foi realizada por apenas um pesquisador e pode acarretar em uma baixa segurança em relação à validade das avaliações feitas.

\item{Como foram extraídos os dados dos estudos primários?}

Após a escolha dos estudos primários, os artigos foram analisados quanto ao seu conteúdo. Resultados quantitativos foram recuperados e analisados. Dados de um mesmo domínio em trabalhos diferentes foram agrupados de forma a facilitar o estudo comparativo das pesquisas. 

Técnicas utilizadas no contexto da pesquisa e suas variações foram documentadas, com seus respectivos resultados e ganhos.

Outras informações relevantes foram extraídas para serem apresentadas de forma sintetizada no trabalho.

\item{Como os dados serão sintetizados?}

Os dados serão tabulados de forma a facilitar a visualização e comparação das abordagens utilizadas nos trabalhos primários consultados. Além disso, as informações mais relevantes, como técnicas utilizadas pela maior parte dos trabalhos, serão apresentadas de forma a demonstrar sua importância, descrevendo detalhes importantes caso mostre-se necessário.

Outras informações relevantes serão apresentadas em forma de resumo na revisão.

\item{Como os dados serão combinados?}

Dados que tratam de abordagens semelhantes serão tabulados de forma conjunta, de forma a facilitar a visualização e comparação das informações.

Análises comparativas poderão ser realizadas, demonstrando as principais técnicas e os melhores resultados em determinados contextos.

Análises quantitativas dos processos descritos nos trabalhos primários também poderão ser realizadas, de forma a identificar possíveis pontos que necessitam de estudos para melhoria dos resultados, bem como para apresentar o estado da arte no domínio da pesquisa.

\end{enumerate}
\bibliographystyle{apalike}
\bibliography{protocolo}

\end{document}